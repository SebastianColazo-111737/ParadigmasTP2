\documentclass[titlepage,a4paper]{article}

\usepackage{a4wide}
\usepackage[colorlinks=true,linkcolor=black,urlcolor=blue,bookmarksopen=true]{hyperref}
\usepackage{bookmark}
\usepackage{fancyhdr}
\usepackage[spanish]{babel}
\usepackage[utf8]{inputenc}
\usepackage[T1]{fontenc}
\usepackage{graphicx}
\usepackage{float}

\pagestyle{fancy} % Encabezado y pie de página
\fancyhf{}
\fancyhead[L]{TP2 - Grupo 05}  % CAMBIAR NOMBRE
\fancyhead[R]{Paradigmas De la Programacion - FIUBA}
\renewcommand{\headrulewidth}{0.4pt}
\fancyfoot[C]{\thepage}
\renewcommand{\footrulewidth}{0.4pt}

\begin{document}
	\begin{titlepage} % Carátula
		\hfill\includegraphics[width=6cm]{logofiuba.jpg}
		\centering
		\vfill
		\Huge \textbf{Trabajo Práctico 2 — Gwent}
		\vskip2cm
		\Large [7507/9502] {Paradigmas De la Programacion}\\
		Primer cuatrimestre de 2025 % Ajustar cuatrimestre
		\vfill
		\begin{tabular}{ | l | l | l | } % Nueva tabla con encabezados
			\hline
			\textbf{Alumno} & \textbf{Número de padrón} & \textbf{Email} \\ \hline
			Sebastian Colazo & & \\ \hline
			Nahuel Giner & 111884 & nginer@fi.uba.ar\\ \hline
			Aksel Mendoza & 108171 & aemendoza@fi.uba.ar \\ \hline
			Miguel Zorrilla & 110619 & mzorrilla@fi.uba.ar  \\ \hline
			Iñaki Vydra & 111505 & ivydra@fi.uba.ar \\ \hline
		\end{tabular}
		\vfill
	\end{titlepage}
	
	\tableofcontents % Índice general
	\newpage
	
	\section{Introducción}\label{sec:intro}
	El presente informe reune la documentación de la solución del primer trabajo práctico de la materia Algoritmos y Programación III que consiste en desarrollar un modelo de predicción de valores de criptomonedas según un divulgadores en Pharo 11 utilizando los conceptos del paradigma de la orientación a objetos vistos hasta ahora en el curso.

	\section{Supuestos}\label{sup:intro}

	\section{Detalles de implementación}\label{sec:implementacion}
	%Deben detallar/explicar qué estrategias utilizaron para resolver todos los puntos más
	%conflictivos del trabajo práctico. Justificar el uso de herencia vs. delegación, mencionar que
	%principio de diseño aplicaron en qué caso y mencionar qué patrones de diseño fueron utilizados y
	%por qué motivos.

	\section{Diagramas de clase}\label{sec:diagramasdeclase}
	% Uno o varios diagramas de clases mostrando las relaciones estáticas entre las clases.  Puede agregarse todo el texto necesario para aclarar y explicar su diseño. Recuerden que la idea de todo el documento es que quede documentado y entendible cómo está implementada la solución.

	\section{Diagramas de secuencia}\label{sec:diagramasdesecuencia}
	% Mostrar las secuencias interesantes que hayan implementado. Pueden agregar texto para explicar si algo no queda claro.

	\section{Diagramas de paquetes}\label{sec:diagramasdepaquetes}
	%Incluir un diagrama de paquetes UML para mostrar el acoplamiento de su trabajo.


\end{document}